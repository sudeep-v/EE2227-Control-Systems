For the feedback transresistance amplifier in Fig. , use small-signal analysis to find the open-loop gain G, the feedback H, and the closed loop gain $G_m$. Neglect $r_o$ of each of the transistors and assume $R_C << \beta_2R_E$ and $R_E << R_F$, and that the feedback causes the signal voltage at the input node to be nearly zero. Evaluate $/frac{V_o}{I_s}$ for the following component values: $\beta_1 = \beta_2 = 100$ , $R_C = R_E = 10 k\ohm$ and $R_F = 100 k\ohm$.

\begin{enumerate}[label=\arabic*.,ref=\theenumi]
%\begin{enumerate}[label=\thesection.\arabic*.,ref=\thesection.\theenumi]
\numberwithin{equation}{enumi}

\item Draw the small-signal equivalent of the circuit in Fig.\ref{fig:ee18btech11045_fb_question}.

\begin{figure}[h!]
	\begin{center}
		\resizebox{\columnwidth}{!}{\begin{circuitikz}[american]
\draw (0,0) node[npn](npn1){Q1}
(npn1.B) -- (-2,0) (-2,-2) to[I , l = $I_{in}$] (-2,0) (-2,-2) node[ground]{}
(npn1.E) -- (0,-1) node[ground]{}
(npn1.C) to[R, l_=$R_{C}$] (0,4) to[short,-o]++(0,0.5) node[right]{$V_{cc}$};

\draw (3,1) node[npn](npn2){Q2}
(npn2.B) -- (0,1) 
(npn2.C) -- (3,4) to[short,-o]++(0,0.5) node[right]{$V_{cc}$}
(npn2.E) to[R, l_ = $R_{E}$] (3,-3) to[short,-o]++(0,-0.5) node[left]{$-V_{ee}$}

(3,0) to[short,-o]++(2,0) node[right]{$V_{out}$}
(4,0) -- (4,-4) to[R, l=$R_{F}$] (-1,-4) -- (-1,0)

;\end{circuitikz}}
	\end{center}
	\caption{}
	\label{fig:ee18btech11045_fb_question}
\end{figure}

\solution

\begin{figure}[h!]
	\begin{center}
		\resizebox{\columnwidth}{!}{\begin{circuitikz}[american]
 \draw (0,0) to[R, l_ = $R_{C}$] (0,3) -- (-1,3)
 (0.5,3)node[above]{$Y$}
 (-1,3)to[I,l_ = $g_{m_1}v_{\pi1}$] (-1,0) -- (0,0)
 
 (-1,0) -- (-3,0) node[ground]{} -- (-4,0) node[left]{$-$}

 (-4,0) to[R, l = $r_{\pi1}$] (-4,3)
 
 (-6,0) node[ground]{} to[I, l = $I_{in}$] (-6,3) -- (-4,3) node[right]{$+$}
 
 (0,3) -- (1.5,3) node[right]{$+$}
 %Add v2 here
 (1.5,3) to[R, l = $r_{\pi2}$] (1.5,0)
 
 (1.5,0) node[left]{$-$} -- (5,0)
 (6,3) node[ground]{} -- (5,3) to[I, l_= $g_{m_2}v_{\pi2}$] (5,0)

(3,0) to[R, l = $R_{E}$] (3,-3) node[ground]{}

(3,-0.5) to[short,-o]++(3,0) node[right]{$V_{out}$}

(4,-0.5)node[above]{$Z$} -- (4,-5) to[R, l=$R_F$] (-5,-5) 
(-5,3)node[above]{$X$} to[short, i_ = $I_F$] (-5,-5) 
;\end{circuitikz}}
	\end{center}
	\caption{}
	\label{fig:ee18btech11045_fb_smallsignalmodel}
\end{figure}

The equivalent circuit is Fig.\ref{fig:ee18btech11045_fb_smallsignalmodel}

\item Find the expression for the open loop Gain(G) of the system.

\solution

The given system is a cascaded system of $Q_1$ and $Q_2$.
The signal flow graph is illustrated in Fig. \ref{fig:ee18btech11045_fb_signalflow}

\begin{figure}[h!]
	\begin{center}
		\resizebox{\columnwidth}{!}{\begin{tikzpicture}

\draw (0,0)node[left]{$I_{in}$} to[short] (2,0)node[above]{X}
(2,0) to[short, *-*, i_ = $G_1$] (4,0) node[above]{Y}
(4,0) to[short, *-*, i_ = $G_2$] (6,0) node[above]{Z}
(6,0) to[short] (8,0)node[right]{$V_{out}$}
(6,0) -- (6,-2) to[short, i = $H$] (2,-2) -- (2,0)

;\end{tikzpicture}
}
	\end{center}
	\caption{}
	\label{fig:ee18btech11045_fb_signalflow}
\end{figure}

So, if the gain of $Q_1$ and $Q_2$ are $G_1$ and $G_2$ respectively, the open-loop gain ($G$) is given by:
\begin{align}
    G = G_1G_2
    \label{eq:ee18btech11045_fb_Gainformula}
\end{align}

$Q_1$ is in CE(Common-emitter) stage. The input signal is $I_{in}$.
From fig. \ref{fig:ee18btech11045_fb_smallsignalmodel},
\begin{align}
    I_X = I_{in}
\end{align}
\begin{align}
    \beta = \frac{I_c}{I_b}
\end{align}
Applying Kirchoff's Law in the loop connecting Y to ground,
\begin{align}
    \implies V_Y = \beta I_{in} R_C
\end{align}
\begin{align}   
    G_1 = \frac{V_{out}}{I_{in}} = \frac{V_{Y}}{I_{X}}
    \\
    =\beta R_c
\end{align}

$Q_2$ is in emitter follower topology.
\begin{align}
    V_{\pi 2}  = V_Y - V_Z
\end{align}
Applying Kirchoff's Law,
\begin{align}
    \frac{V_Y - V_Z}{r_{\pi}} + g_{m2}\brak{V_Y-V_Z} = \frac{V_Z}{R_E}
\end{align}
\begin{align}
    \implies \frac{V_Z}{V_Y} = \frac{R_E}{\frac{1}{g_{m2}} + R_E}
\end{align}
\begin{align}
    \implies G_2 = \frac{R_E}{\frac{1}{g_{m2}} + R_E}
\end{align}

From \eqref{eq:ee18btech11045_fb_Gainformula}, the open loop gain ($G$):
\begin{align}
    G = \brak{\beta_1 R_c} \brak{\frac{R_E}{\frac{1}{g_{m2}} + R_E}}
    \label{eq:ee18btech11045_fb_Gain}
\end{align}

\item Find the feedback factor(H) of the given circuit.

\solution

From Fig.\ref{fig:ee18btech11045_fb_smallsignalmodel}, the feedback circuit consists of only a resistor $R_F$:
\begin{align}
    \therefore H = \frac{I_F}{V_{out}} = \frac{1}{R_F}
    \label{eq:ee18btech11045_fb_feedbackfactor}
\end{align}

\item Find the closed loop gain of the system.

\solution 

The closed loop gain of a system is given by:
\begin{align}
    G_L = \frac{G}{1+GH}
\end{align}

From \eqref{eq:ee18btech11045_fb_Gain} and \eqref{eq:ee18btech11045_fb_feedbackfactor}. The closed loop gain of the circuit is given by:
\begin{align}
    G_L = \frac{\brak{\beta_1 R_c} \brak{\frac{R_E}{\frac{1}{g_{m2}} + R_E}}}{1 + \frac{\brak{\beta_1 R_c} \brak{\frac{R_E}{\frac{1}{g_{m2}} + R_E}}}{R_F}}
    \\
    = \frac{R_FR_CR_E\beta}{\beta R_C R_E + R_F\brak{\frac{1}{g_{m2}}+R_E}}
    \label{eq:ee18btech11045_fb_clgain}
\end{align}


\item Find G,H and $G_L$ for the given problem. Parameters are summarised in table \ref{table:ee18btech11045_table1}.
%
\begin{table}[!ht]
\centering
\def\ifundefined#1{\expandafter\ifx\csname#1\endcsname\relax}

\ifundefined{inputGnumericTable}

	\def\gnumericTableEnd{\end{document}}


%%%%%%%%%%%%%%%%%%%%%%%%%%%%%%%%%%%%%%%%%%%%%%%%%%%%%%%%%%%%%%%%%%%%%%
%%                                                                  %%
%%  This is the PREAMBLE. Change these values to get the right      %%
%%  paper size and other niceties.                                  %%
%%                                                                  %%
%%%%%%%%%%%%%%%%%%%%%%%%%%%%%%%%%%%%%%%%%%%%%%%%%%%%%%%%%%%%%%%%%%%%%%

	\documentclass[12pt%
			  %,landscape%
                    ]{report}
       \usepackage[latin1]{inputenc}
       \usepackage{fullpage}
       \usepackage{color}
       \usepackage{array}
       \usepackage{longtable}
       \usepackage{calc}
       \usepackage{multirow}
       \usepackage{hhline}
       \usepackage{ifthen}
%%  End of the preamble for the standalone. The next section is for %%
%%  documents which are included into other LaTeX2e files.          %%
\else

%%  We are not a stand alone document. For a regular table, we will %%
%%  have no preamble and only define the closing to mean nothing.   %%
    \def\gnumericTableEnd{}

%%  If we want landscape mode in an embedded document, comment out  %%
%%  the line above and uncomment the two below. The table will      %%
%%  begin on a new page and run in landscape mode.                  %%
%       \def\gnumericTableEnd{\end{landscape}}
%       \begin{landscape}


%%  End of the else clause for this file being \input.              %%
\fi

%%%%%%%%%%%%%%%%%%%%%%%%%%%%%%%%%%%%%%%%%%%%%%%%%%%%%%%%%%%%%%%%%%%%%%
%%                                                                  %%
%%  The rest is the gnumeric table, except for the closing          %%
%%  statement. Changes below will alter the table's appearance.     %%
%%                                                                  %%
%%%%%%%%%%%%%%%%%%%%%%%%%%%%%%%%%%%%%%%%%%%%%%%%%%%%%%%%%%%%%%%%%%%%%%

\providecommand{\gnumericmathit}[1]{#1} 
%%  Uncomment the next line if you would like your numbers to be in %%
%%  italics if they are italizised in the gnumeric table.           %%
%\renewcommand{\gnumericmathit}[1]{\mathit{#1}}
\providecommand{\gnumericPB}[1]%
{\let\gnumericTemp=\\#1\let\\=\gnumericTemp\hspace{0pt}}
 \ifundefined{gnumericTableWidthDefined}
        \newlength{\gnumericTableWidth}
        \newlength{\gnumericTableWidthComplete}
        \newlength{\gnumericMultiRowLength}
        \global\def\gnumericTableWidthDefined{}
 \fi
%% The following setting protects this code from babel shorthands.  %%
 \ifthenelse{\isundefined{\languageshorthands}}{}{\languageshorthands{english}}
%%  The default table format retains the relative column widths of  %%
%%  gnumeric. They can easily be changed to c, r or l. In that case %%
%%  you may want to comment out the next line and uncomment the one %%
%%  thereafter                                                      %%
\providecommand\gnumbox{\makebox[0pt]}
%%\providecommand\gnumbox[1][]{\makebox}

%% to adjust positions in multirow situations                       %%
\setlength{\bigstrutjot}{\jot}
\setlength{\extrarowheight}{\doublerulesep}

%%  The \setlongtables command keeps column widths the same across  %%
%%  pages. Simply comment out next line for varying column widths.  %%
\setlongtables

\setlength\gnumericTableWidth{%
	60pt+%
	100pt+%
0pt}
\def\gumericNumCols{2}
\setlength\gnumericTableWidthComplete{\gnumericTableWidth+%
         \tabcolsep*\gumericNumCols*2+\arrayrulewidth*\gumericNumCols}
\ifthenelse{\lengthtest{\gnumericTableWidthComplete > \linewidth}}%
         {\def\gnumericScale{\ratio{\linewidth-%
                        \tabcolsep*\gumericNumCols*2-%
                        \arrayrulewidth*\gumericNumCols}%
{\gnumericTableWidth}}}%
{\def\gnumericScale{1}}

%%%%%%%%%%%%%%%%%%%%%%%%%%%%%%%%%%%%%%%%%%%%%%%%%%%%%%%%%%%%%%%%%%%%%%
%%                                                                  %%
%% The following are the widths of the various columns. We are      %%
%% defining them here because then they are easier to change.       %%
%% Depending on the cell formats we may use them more than once.    %%
%%                                                                  %%
%%%%%%%%%%%%%%%%%%%%%%%%%%%%%%%%%%%%%%%%%%%%%%%%%%%%%%%%%%%%%%%%%%%%%%

\ifthenelse{\isundefined{\gnumericColA}}{\newlength{\gnumericColA}}{}\settowidth{\gnumericColA}{\begin{tabular}{@{}p{60pt*\gnumericScale}@{}}x\end{tabular}}
\ifthenelse{\isundefined{\gnumericColB}}{\newlength{\gnumericColB}}{}\settowidth{\gnumericColB}{\begin{tabular}{@{}p{100pt*\gnumericScale}@{}}x\end{tabular}}
\begin{tabular}[c]{%
	b{\gnumericColA}%
	b{\gnumericColB}%%
	}



\hhline{|-|-}
	 \multicolumn{1}{|p{\gnumericColA}|}%
	{\gnumericPB{\centering}\textbf{Parameters}}
	&\multicolumn{1}{p{\gnumericColB}|}%
	{\gnumericPB{\centering}\textbf{Value}}

\\
\hhline{|--|}
	 \multicolumn{1}{|p{\gnumericColA}|}%
	{\gnumericPB{\centering}$V_{cc}$}
	&\multicolumn{1}{p{\gnumericColB}|}%
	{\gnumericPB{\centering}$5 V$}
	

\\
\hhline{|--|}
	 \multicolumn{1}{|p{\gnumericColA}|}%
	{\gnumericPB{\centering}$\beta_1$}
	&\multicolumn{1}{p{\gnumericColB}|}%
	{\gnumericPB{\centering}$100$}

\\
\hhline{|--|}
	 \multicolumn{1}{|p{\gnumericColA}|}%
	{\gnumericPB{\centering}$\beta_2$}
	&\multicolumn{1}{p{\gnumericColB}|}%
	{\gnumericPB{\centering}$100$}

\\
\hhline{|--|}
	 \multicolumn{1}{|p{\gnumericColA}|}%
	{\gnumericPB{\centering}$R_{C}$}
	&\multicolumn{1}{p{\gnumericColB}|}%
	{\gnumericPB{\centering}$10K\ohm$}

\\
\hhline{|--|}
	 \multicolumn{1}{|p{\gnumericColA}|}%
	{\gnumericPB{\centering}$R_{E}$}
	&\multicolumn{1}{p{\gnumericColB}|}%
	{\gnumericPB{\centering}$10K\ohm$}
	
\\
\hhline{|--|}
	 \multicolumn{1}{|p{\gnumericColA}|}%
	{\gnumericPB{\centering}$R_{F}$}
	&\multicolumn{1}{p{\gnumericColB}|}%
	{\gnumericPB{\centering}$100K\ohm$}

\\
\hhline{|-|-|}
\end{tabular}

\ifthenelse{\isundefined{\languageshorthands}}{}{\languageshorthands{\languagename}}
\gnumericTableEnd
\caption{}
\label{table:ee18btech11045_table1}
\end{table}

\solution

To calculate the bias values of Q1, Q2. Remove the input and output, the resultant circuit is shown in fig.\ref{fig:biascalc}

\begin{figure}[!ht]
	\begin{center}
			\resizebox{\columnwidth}{!}{\begin{circuitikz}[american]
\draw (0,0) node[npn](npn1){Q1}
(-1,0) to[short, i = $I_{b1}$] (npn1.B)
(npn1.E) -- (0,-1) node[ground]{}
(0,4) to[R, l_=$R_{C}$, i_ = $\beta I_{b1}$ ] (npn1.C)
(0,4) to[short,-o]++(0,0.5) node[right]{$V_{cc}$};

\draw (3,1) node[npn](npn2){Q2}
(0,1) to[short, i = $I_{b2}$] (npn2.B)
(npn2.C) -- (3,4) to[short,-o]++(0,0.5) node[right]{$V_{cc}$}
(npn2.E) to[short, i_ = $(\beta+1)I_{b2}$ ] (3,0) to[R, l_ = $R_{E}$, i_ = $(\beta+1)I_{b2} + I_{b_1}$] (3,-3) to[short,-o]++(0,-0.5) node[left]{$-V_{ee}$}

(3,0) -- (4,0) -- (4,-4) to[R, l=$R_{F}$] (-1,-4) -- (-1,0)

;\end{circuitikz}}
	\end{center}
\caption{}
\label{fig:biascalc}
\end{figure}

Applying KVL to the circuit, we get:
\begin{align}
    0.7  + I_{b1}R_F + \brak{I_{b1} - (\beta+1)I_{b2}}R_E = -V_{ee}
\end{align}
\begin{align}
    0.7 + I_{b1}R_F + 0.7 + \brak{\beta I_{b1} + I_{b2}}R_C = V_{cc}
\end{align}

Solving the above equations, we get:
\begin{align}
    I_{b1} = \frac{\frac{V_{cc} - 1.4}{R_C} - \frac{V_{ee} +0.7}{R_E(\beta+1)}}{\frac{R_F + \beta R_C}{R_C} + \frac{R_E + R_F}{R_E(\beta + 1)}}
    \\
    = 3.22 * 10^{-6}
\end{align}
\begin{align}
    I_{b2} = \frac{\frac{V_{cc} - 1.4}{\beta R_C + R_F} + \frac{V_{ee}+0.7}{R_E+R_F}}{\frac{R_C}{R_F+\beta R_C} + \frac{R_E(\beta+1)}{R_E+R_F}}
\end{align}

we know, 
\begin{align}
    g_m = \frac{I_{c}}{V_T}
\end{align}
where, $V_T$ = 26mV, and
\begin{align}
    r_{\pi} = \frac{\beta}{g_m}
\end{align}

\begin{align}
    \therefore g_{m1} &= \frac{\beta I_{b1}}{V_T}
    \\
    &= 0.0123
\end{align}
\begin{align}
    r_{\pi 1} &= \frac{\beta}{g_{m1}}
    \\
    &= 8130 \ohm
\end{align}
\begin{align}    
    \therefore g_{m2} &= \frac{\beta I_{b2}}{V_T}
    \\
    &= 0.023
\end{align}
\begin{align}
    r_{\pi 2} &= \frac{\beta}{g_{m2}}
    \\
    &= 4347.8 \ohm
\end{align}

From \eqref{eq:ee18btech11045_fb_Gain}, the open loop gain (G):
\begin{align}
    G &= \brak{100*(10K)}\brak{\frac{10^{4}}{10^{4} + \frac{1}{0.023}}} \ohm
    \\
    &= \brak{10^{6}}\brak{0.995}
    \\
    &= 995670 \ohm
\end{align}

From \eqref{eq:ee18btech11045_fb_feedbackfactor}, the feedback (H):
\begin{align}
    H = \frac{1}{100K} \ohm^{-1}
    \implies H = 10^{-5} \ohm^{-1}
\end{align}

From \eqref{eq:ee18btech11045_fb_clgain}, the closed loop gain ($G_L$):
\begin{align}
    G_L &= \frac{995670}{1 + (995670)(10^{-5})} \ohm
    \\
    &= 99006.52 \ohm
    \label{eq:ee18btech11045_closedloopgaincl}
\end{align}

\item Verify the result using spice simulation.

\solution

The following netlist simulates the closed loop gain for a sinusoidal signal of amplitude $10^{-6}$
\begin{lstlisting}
codes/ee18btech11045/spice/ee18btech11045_clresult.net
\end{lstlisting}

The output is plotted using the folowing code.
\begin{lstlisting}
codes/ee18btech11045/spice/ee18btech11045_clresult.py
\end{lstlisting}

The output is plotted in fig. \ref{fig:ee18btech11045_fb_clresult}. The output amplitude is shown to be 0.1 .
\begin{align}
    \therefore \frac{V_{out}}{I_{in}} \approx 10^{5}
\end{align}
This proves the value calculated in \eqref{eq:ee18btech11045_closedloopgaincl}.

\begin{figure}[h!]
	\begin{center}
		\resizebox{\columnwidth}{!}{\input{./figs/ee18btech11045/ee18btech11045_clresult.eps}}
	\end{center}
	\caption{}
	\label{fig:ee18btech11045_fb_clresult}
\end{figure}

\item Represent the circuit using a Feedback Block diagram.

\solution

\begin{figure}[!ht]
	\begin{center}
			\resizebox{\columnwidth}{!}{\begin{circuitikz}[american]
\usetikzlibrary{positioning, fit, calc}
\draw (0,0) to[I = $I_{in}$, *-*] (0,2) to[short, *-*] (2,2)node[above]{$+$} to[R=$R_{in}$,*-*] (2,0)node[below]{$-$}
(2-0.2,1) node[left]{$V_{in}$}
(2,2) to[short, *-*] (5,2) {}
(7,1)node[draw,minimum width=4cm,minimum height=4cm] (load) {Gain Amplifier}{}
(7,-4)node[draw,minimum width=4cm,minimum height=4cm] (load) {Feedback Network}{}
(0,0) to[short, *-*] (5,0)
(14,2) to[R = $R_{out}$, *-*] (9,2)
(14,2) node[right]{$+$}
(14,1) node[]{$V_{out}$}
(14,0) node[right]{$-$}
(14,0) to[short, *-*] (9,0)
(3,0) to[short, *-*] (3,-5) to[short, *-*] (5,-5){}
(4,2) to[short, *-*, i = $I_F$] (4,-3)
(4,-3) to[short, *-*] (5,-3){}
(10,2) to[short, *-*] (10,-3) to[short, *-*] (9,-3){}
(9,-5) to[short, *-*] (11,-5) to[short, *-*] (11,0){}
;\end{circuitikz}}
	\end{center}
\caption{Shunt-Shunt Feedback}
\label{fig:Block_Diagram}
\end{figure}

\item Calculate the input resistance of the open loop and closed loop system and compare.

\solution

For the cascaded system of $Q1$ and $Q2$, the input resistance of the system $R_{in}$,
\begin{align}
    R_{in} = R_{in_{Q1}}   
\end{align}

To calculate the input resistance of the system, shot the independent sources  and find the ratio $\frac{V_in}{I_in}$. As, $Q1$ is in common-emitter stage, from the fig. \ref{fig:ee18btech11045_fb_smallsignalmodel},
\begin{align}
    R_in &= r_{\pi1}
    \\
    &= 8130 \ohm
\end{align}

After feedback is applied, to calculate input resistance, consider an input source $V_{in}$ is connected to the input,
\begin{align}
    V_{out} &= I_{in}G
    &= \frac{V_{in}}{R_{in}}G
\end{align}
But as feedback is applied,
\begin{align}
    V_{in} &= V_{out} H
    &= \frac{V_{in}}{R_{in}}GH
\end{align}
Applying KVL for input loop in fig. \ref{fig:Block_Diagram},
\begin{align}
    \brak{I_{in} - \frac{V_{in}}{R_{in}}GK}R_{in} = V_{in}
\end{align}
\begin{align}
    \implies R_{in_{cl}} = \frac{V_{in}}{I_{in}} = \frac{R_in}{1+GH}
\end{align}
where, $R_in$ is the input resistance of open loop.Therefore, $R_{in}$ after feedback:
\begin{align}
    R_{in} &= \frac{r_{pi1}}{1 + GH}
    \\
    &= \frac{8130}{10.95} = 742 \ohm
\end{align}

The input should act as an ideal current source, so as the input resistance is decreased, the feedback gives more favourable value of $R_{in}$.

\item Calculate the output resistance of the open loop and closed loop system and compare.

\solution

Similar to the input resistance, the output resistance of the cascaded system is the output resistance of $Q2$. As $Q2$ is in emitter follower configuration, from the fig. \ref{fig:ee18btech11045_fb_smallsignalmodel},

\begin{align}
    R_{out} &= \frac{1}{g_{m2}}
    \\
    &= \frac{1}{0.023} = 43.37 \ohm
\end{align}

To calculate output resistance after feedback is applied, consider a voltage source $V_X$ applied at $V_{out}$ with output current $I_X$ :
\begin{align}
    V_{in} = HV_{X}
\end{align}
\begin{align}
    \implies V_{out} = GHV_{X}
\end{align}
Applying KVL at ouput loop,
\begin{align}
    \frac{GHV_X+V_X}{R_{out}} = I_X
\end{align}
The closed loop output impedance,
\begin{align}
    R_{out_{cl}} &= \frac{V_X}{I_X}
    \\
    &= \frac{R_{out}}{1 + GH}
\end{align}
The output resistance after feedback:
\begin{align}
    R_{out} = \frac{43.37}{10.95} = 3.96 \ohm
\end{align}

The ouput should act as an ideal voltage source, i.e the output resistance should be as low as possible. As feedback reduces the values of $R_{out}$, it causes the output resistances to be more favourable.

\end{enumerate}